%-*-coding: utf-8-*-
\FloatBarrier
\chapter{Тестирование и результаты}

Для тестирования предложенного подхода была написана программа PhyloSAT, реализующая описанное сведение.
Для написания использовался язык программирования Java 7.
Программа собирается в JAR-файл, не требует никаких зависимостей и может выполняться на любой операционной системе с установленной средой выполнения Java.
Кроме минимально необходимых параметров для указания входных и выходных файлов, поддерживаются также параметры, позволяющие выполнить поиск сети с фиксированным гибридизационным числом.
Это, в частности, позволяет использовать PhyloSAT для построения конкретных гибридизационных сетей, или для доказательства того, что таких сетей не существует.

\FloatBarrier
\section{Описание экспериментов}

Для экспериментального сравнения использовался реальный набор данных, соответствующий злаковым травам~\cite{grass2001phylogeny}.
Набор состоит из 57 тестов.
Каждый тест содержит от 2 до 6 деревьев и от 5 до 47 таксонов.
Все эксперименты проводились на компьютере с процессором AMD Phenom II X6 1090T 3.2 GHz и операционной системой Ubuntu 14.04.
Каждый тест запускался с лимитом по времени 1000 секунд.
Для сравнения производительности в таких же условиях запускались программы $\mathrm{PIRN_{CH}}$ и $\mathrm{PIRN_C}$.
Неточная версия алгоритма реализована следующим образом: по истечению заданного лимита времени выводится наилучший найденный ответ.

\FloatBarrier
\section{Результаты}

Тестовый набор состоит из различных по сложности тестов.
Так, например, 9 тестов не были решены ни одним из алгоритмов за отведенной время.
С другой стороны многие тесты имели тривиальное решение, и все тестируемые алгоритмы выдавали моментальный ответ.

\subsection{Точный алгоритм}

PIRN$\mathrm{_C}$ смог решить 29 тестов из 57.
Оказалось, что во всех решенных тестах гибридизационное число минимальной сети не превосходит 5.
Следовательно, PIRN$\mathrm{_C}$ не подходит для построения гибридизационных сетей с большим гибридизационным числом.

PhyloSAT смог решить на 7 тестов больше, чем PIRN$\mathrm{_C}$.
Кроме того, PhyloSAT оказался быстрее чем PIRN$\mathrm{_C}$ на всех тестах, на которых были получены ответы.
Результаты тестирования точных алгоритмов кратко суммированы в Таблице~\ref{exact-results-table}.

\begin{table}[t]
\caption{Результаты тестирования точных алгоритмов.}
\centering
\begin{tabular}{l | l}
	Алгоритм & Решено тестов \\
	\hline
	PhyloSAT & 36 \\
	PIRN$\mathrm{_C}$ & 29 \\
\end{tabular}
\label{exact-results-table}
\end{table}

\subsection{Неточный алгоритм}

Из описания реализации понятно, что неточная версия PhyloSAT способна решить все те тесты, что и точная версия, плюс выдает какие-то, возможно не минимальные, гибридизационные сети для остальных тестов.
PhyloSAT смог решить 48 тестов, что на 5 тестов больше, чем PIRN$\mathrm{_{CH}}$.
По времени работы явного лидера нет, так как некоторые тесты PIRN$\mathrm{_{CH}}$ выполняет быстрее, чем PhyloSAT.
Результаты тестирования неточных алгоритмов суммированы в Таблице~\ref{heuristic-results-table}.

\begin{table}[t]
\caption{Результаты тестирования неточных алгоритмов.}
\centering
\begin{tabular}{l | l}
	Алгоритм & Решено тестов \\
	\hline
	PhyloSAT & 48 \\
	PIRN$\mathrm{_{CH}}$ & 43 \\
\end{tabular}
\label{heuristic-results-table}
\end{table}

Наиболее интересными являются различия между неточными алгоритмами на тех тестах, которые не смогли решить точные алгоритмы.
Всего таких тестов 12.
В трех случаях PhyloSAT нашел более оптимальную сеть, в двух случаях менее оптимальную, а в оставшихся семи случаях алгоритмы представили одинаковые сети.
В двух случаях PhyloSAT оказался быстрее и медленнее в пяти случаях.
Краткое сравнение этих тестов показано в Таблице~\ref{results-comparison-table}.
Подробная таблица с результатами тестирования PhyloSAT и PIRN$\mathrm{_{CH}}$ приведена в Приложении~\ref{AppendixB}.

\begin{table}[t]
\caption{Сравнение результатов неточных алгоритмов на 12 тяжелых тестах.}
\centering
\begin{tabular}{l | l | l}
	Алгоритм & Лучший результат & Лучшее время \\
	\hline
	PhyloSAT & 3 & 2 \\
	PIRN$\mathrm{_{CH}}$ & 2 & 5 \\
\end{tabular}
\label{results-comparison-table}
\end{table}

Заметим, что во многих тестах оценка нижней границы гибридизационного числа, которое предоставляет PIRN$\mathrm{_{CH}}$ реализуется и является фактическим минимумом.
Сравнение гибридизационных чисел найденных сетей с соответствующими оценками нижних границ приведены в Таблице~\ref{lower-bound-table}.
В 4 из 12 тестов PhyloSAT нашел оптимальную сеть, но не успел доказать ее минимальность за отведенное время, так как, как упоминалось выше, наибольшее время занимает поиск сети с гибридизационным числом на единицу меньшим минимального.
Из этого наблюдения следует, что время работы алгоритма значительно уменьшится в тех случаях, когда нижняя оценка гибридизационного числа совпадает с реальным минимумом, так как не надо будет доказывать его минимальность.

\begin{table}[t]
\caption{Сравнение полученных гибридизационных чисел с теоретической нижней границей, предоставленной алгоритмом PIRN}
\centering
\begin{tabular}{l | l | l}
	Тест & Полученный ответ & Нижняя граница \\
	\hline
	RbclRpoc & \underline{7} & \underline{7} \\
	NdhfPhytIts & 13 & 11 \\
	NdhfPhytRpoc & 8 & 6 \\
	NdhfRbclRpoc & 12 & 10 \\
	NdhfWaxyIts & 8 & 7 \\
	PhytRbclIts & 9 & 7 \\
	PhytRpocIts & \underline{7} & \underline{7} \\
	RbclWaxyIts & \underline{6} & \underline{6} \\
	NdhfPhytRbclRpoc & 9 & 7 \\
	NdhfPhytRpocIts & 10 & 7 \\
	NdhfRbclWaxyIts & \underline{6} & \underline{6} \\
	PhytRbclRpocIts & 9 & 6 \\
\end{tabular}
\label{lower-bound-table}
\end{table}

Примеры некоторых входных деревьев и построенных для них минимальных сетей представлены в Приложении~\ref{AppendixA}.
