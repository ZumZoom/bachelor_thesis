%-*-coding: utf-8-*-
\FloatBarrier
\chapter{Обзор предметной области}

\FloatBarrier
\section{Основные определения}

\subsection{Филогенетика}

\emph{Филогенетика} --- область биологической систематики, которая занимается идентификацией и прояснением эволюционных взаимоотношений среди разных видов жизни на Земле~\cite{wiki:phylogenetics}.

\emph{Филогенетическим деревом} называется дерево, отражающее эволюционные взаимосвязи между различными видами или другими сущностями, имеющими общего предка~\cite{wiki:phylogenetic-tree}.
В данной работе будут рассматриваться укорененные двоичные деревья. Листья филогенетического дерева отображают таксоны, а узлы представляют из себя эволюционные события: разделение предкового вида на два независимых.

\emph{Гибридизационная сеть} --- направленный ациклический граф с выделенным корнем.
Гибридизационная сеть состоит из трех типов вершин - обычных вершин, ретикулярных вершин и листьев.
Обычные вершины имеют одного предка и нескольких потомков.
\emph{Ретикулярные вершины} имеют более одного предка и нескольких потомков.
Листья имеют одного предка и не имеют потомков.
В данной работе, без потери общности, будет рассматриваться упрощенная модель гибридизационной сети, в которой обычные вершины имеют ровно двух потомков, а ретикулярные вершины имеют ровно двух предков и ровно одного потомка. 
Заметим, что привести произвольную гибридизационную сеть к упрощенному виду не составляет труда~\cite{wu2010close}.

Гибридизационную сеть можно свести к филогенетическому дереву следующим образом:

\begin{itemize}
	\item У каждой ретикулярной вершины необходимо выбрать используемого предка, а ребро ведущее к другому предку удалить.
	\item Стянуть все ребра, имеющие ровно одного предка и ровно одного потомка.
\end{itemize}

Гибридизационная сеть $N$ содержит в себе дерево $T$, если существует такой способ выбора удаляемых ребер, что после стягивания получится дерево $T'$ изоморфное дереву $T$.

\emph{Гибридизационным числом} сети $N$ с корнем $\rho$ называется величина $h(N) = \sum\limits_{v \ne \rho} (d^-(v) - 1)$, где за $d^-(v)$ обозначена входящая степень вершины $v$.
Аналогично обозначим за $d^+(v)$ исходящую степень вершины $v$.
Заметим, что в нашей модели значение $h(N)$ равно количеству ретикулярных вершин.

Рассмотрим множество из $k$ филогенетических деревьев $T_1, T_2, \dots, T_k$, построенных над фиксированным множеством таксонов.
Сеть $N_{min}$ называется \emph{минимальной гибридизационной сетью}, если она принадлежит множеству сетей, содержащих в себе все $k$ деревьев, и при этом имеет наименьшее возможное гибридизационное число.

Множество таксонов $A$ называется \emph{кластером} на деревьях $T_1, T_2, \dots, T_k$, если в каждом дереве $T_i$ существует вершина $v_i$, такая, что множество листьев в поддереве $v_i$ совпадает с множеством $A$.

\subsection{Выполнимость булевой формулы}

\emph{Задача о выполнимости булевой формулы (SAT)} заключается в следующем: можно ли назначить всем переменным, встречающимся в формуле, значения ложь и истина так, чтобы формула стала истинной~\cite{wiki:sat}. Обычно рассматриваются формулы в конъюктивной нормальной форме.

\emph{SAT-солвер} --- программа предназначенная для решения задачи выполнимости булевой формулы.

\FloatBarrier
\section{Постановка задачи}

Задача построения минимальной гибридизационной сети заключается в том, чтобы для множества филогенетических деревьев $T_1, T_2, \dots, T_k$ построить минимальную гибридизационную сеть.

Было показано, что даже для случая двух деревьев эта задача NP-полна~\cite {bordewich2007computing}.

\FloatBarrier
\section{Обзор существующих методов}
