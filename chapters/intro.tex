% -*-coding: utf-8-*-
\startprefacepage

Построение минимальной филогенетической сети является важной задачей филогенетики.
Филогенетическая сеть используется для представления эволюционных взаимосвязей между различными биологическими видами в ретикулярной модели эволюции.
В ретикулярной (сетчатой) модели эволюции, в отличие от классической (древовидной) модели, присутствует горизонтальный перенос генов, а также гибридизация между видами, что не позволяет использовать более простые структуры, такие как филогенетические деревья.
Филогенетические сети активно исследуются в настоящее время~\cite{huson2010phylogenetic, morrison2011introduction, 
nakhleh2011evolutionary}.

Хотя существует несколько формальных определений филогенетических сетей, в данной работе рассматриваются только гибридизационные сети~\cite{semple2006hybridization, chen2010hybridnet}.
Исходными данными для построения гибридизационной сети является набор из нескольких филогенетических деревьев, построенных на одном и том же множестве таксонов.
Каждое филогенетическое дерево представляет собой эволюционную историю какого-то гена.
Деревья могут иметь различную топологию из-за наличия ретикуляций.
Задача состоит в том, чтобы построить гибридизационную сеть с минимальным количеством вершин, содержащую в себе каждое из исходных деревьев как подграф.

Большинство алгоритмов для построения гибридизационных сетей основаны на эвристиках и не гарантируют точный результат~\cite{wu2010close, park2012murpar}.
Кроме того, многие алгоритмы не могут работать более чем с двумя деревьями.
Недавно был представлен точный алгоритм, работающий с многими деревьями~\cite{wu2013algorithm}.
С точной и эвристической версией этого алгоритма, как с наиболее производительной на текущий момент, будет производиться сравнение.

В данной работе представлен новый способ построения минимальной гибридизационной сети, основанный на сведении к задаче о выполнимости булевой формулы (SAT), гарантирующий точный результат.
На основе представленного алгоритма была разработана утилита PhyloSAT, решающая поставленную задачу. Исходный код утилиты доступен на GitHub\footnote{\url{https://github.com/ctlab/PhyloSAT}}.

Подходы, основывающиеся на сведении к задаче SAT, успешно применяются для решения различных задач, в том числе для задач филогенетики~\cite{bonet2009efficiently}, построения конечных автоматов~\cite{heule2010exact} и верификации~\cite{biere2003bounded}.
Это обусловлено тем, что современные SAT-солверы хорошо оптимизированы, и способны успешно справляться с формулами из десятков тысяч переменных за несколько минут.
